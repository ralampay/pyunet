\documentclass[acmsmall]{acmart}
\begin{document}

\title{An Robustness Evaluation of UNet Configurations for Semantic Segmentations of Covid19 CT Scans}

\author{Raphael Alampay}
\email{ralampay@ateneo.edu}
\affiliation{%
  \institution{Ateneo de Manila University}
  \country{Philippines}
}

%%
%% The abstract is a short summary of the work to be presented in the
%% article.
\begin{abstract}
  TODO: ABSTRACT
\end{abstract}

%%
%% Keywords. The author(s) should pick words that accurately describe
%% the work being presented. Separate the keywords with commas.
\keywords{covid19, machine learning, unet, segmentation}

%% A "teaser" image appears between the author and affiliation
%% information and the body of the document, and typically spans the
%% page.
\begin{teaserfigure}
  \includegraphics[width=\textwidth]{sampleteaser}
  \caption{Semantic Segmentation of Covid19 CT Scans}
  \Description{Qualitative Results of Semantic Segmentation using the UNet Architecture}
  \label{fig:teaser}
\end{teaserfigure}

%%
%% This command processes the author and affiliation and title
%% information and builds the first part of the formatted document.
\maketitle

\section{Introduction}

TODO: INTRODUCTION

\section{Review of Related Literature}

\subsection{Semantic Segmentation}

\subsection{UNet}

\subsection{Double Convolution}

\subsection{Normalization Techniques}

\subsection{Non-linear Activations}

\section{Methodology}

\section{Experimental Results}

\section{Analysis}

\section{Conclusion}

\section{Bibliography}

\section{Acknowledgments}

To follow.
%%
%% The next two lines define the bibliography style to be used, and
%% the bibliography file.
\bibliographystyle{ACM-Reference-Format}
\bibliography{sample-base}

\end{document}
\endinput
